\documentclass[handout]{beamer}
\usepackage[utf8]{inputenc}
\usetheme{default}
\usecolortheme{dove}
\usepackage{textpos}
\usepackage{grid-system}

% \usepackage[rm]{roboto}
\usepackage[T1]{fontenc}
%\usepackage[sfdefault]{roboto}

% po4a: environment frame
% po4a: environment Row
% po4a: environment Cell

\AtBeginSection[] % Do nothing for \section*
{
	\begin{frame}<beamer>
		\frametitle{Übersicht}
		\tableofcontents[currentsection]
	\end{frame}
}

\title{\centering\includegraphics[width=\textwidth]{images/logo-skyline}}
\author{}
%\institute[Inst.]{Freifunk Darmstadt}
\date{\footnotesize 4. Oktober 2015}

\begin{document}


\begin{frame}
\maketitle
\end{frame}

\addtobeamertemplate{frametitle}{}{%
\begin{textblock*}{100mm}(0.92\textwidth,-0.5cm)
\includegraphics[height=1cm]{images/logo}
\end{textblock*}}

\begin{frame}{Übersicht}
\tableofcontents
\end{frame}

\section{Was ist Freifunk?}
\begin{frame}
	\frametitle{Was ist Freifunk?}

	Deutschlandweite Initiative für freie WLAN-Netze
	
	\pause
	\vfill
	\centering
	\includegraphics[scale=0.3]{images/2015-10_freifunk-map} \\
	über 200 lokale Gruppen\\bundesweit ca. 20.000 Accesspoints

	
\end{frame}

\begin{frame}
	\frametitle{Wie sieht so ein freies Netz aus?}
	\begin{center}
		\includegraphics[height=6cm]{images/2015-08-31_starkenburg}
	\end{center}
\end{frame}

\begin{frame}
	\frametitle{Wie sieht so ein freies Netz aus?}
	\begin{center}
		\includegraphics[height=6cm]{images/2015-10_partenheim-map}
	\end{center}
\end{frame}

\begin{frame}{Was kann man damit tun?}
	\vfill
	\begin{center}
		\includegraphics[width=5.5cm]{images/verbindet}
	\end{center}
	
	\begin{itemize}[<+->]
		\item jeder kann Services anbieten und nutzen
		\begin{itemize}
			\item Telefonieren und Chatten
			\item dezentrales Social-Media (z.B Diaspora, Twister)
			\item lizenzfreies Community-Radio
			\item Austausch von Dateien und Medien
			\item \ldots
		\end{itemize}
		\item gemeinsame Nutzung eines Internetanschlusses
		\item Testbed f\"ur wissenschaftliche Experimente
	\end{itemize}
	\vfill
\end{frame}

\begin{frame}
	\frametitle{Vision}
	
	Wie soll unsere Kommunikationsinfrastruktur aussehen?
	
	\begin{itemize}[<+->]
		\item öffentlich
		\item anonym zugänglich
		\item nicht kommerziell
		\item unzensiert
		\item dezentral organisiert
		\item im Besitz einer Gemeinschaft
	\end{itemize}
\end{frame}

\begin{frame}
	\frametitle{Vision: offen und öffentlich}
	
	\begin{itemize}[<+->]
		\item freie, ungehinderte Teilnahme an Betrieb und Ausbau des Netzes $\rarrow$ Mitmachnetz!
		\item freier Zugang zu Netz und Diensten (keine Unterscheidung nach Ort oder Geldbeutel)
	\end{itemize}
	% Bilder: Zugänglichkeit, Menschengruppe
\end{frame}

\begin{frame}
	\frametitle{Vision: anonyme Zugänglichkeit}
	
	\begin{itemize}[<+->]
		\item Das bedeutet nicht, dass Freifunk Netze den Datenverkehr anonymisieren.
		\item Achtung: Freifunk ist kein Anonymisierungsdienst (wie z.B. Tor), Zuordnung der Daten zu Person deutlich einfacher möglich!
		\item (Verschlüsselung nicht vergessen)
	\end{itemize}
	% Bilder: Tor, Schlüssel
\end{frame}

\begin{frame}
	\frametitle{Vision: Dezentralität}
	
	\begin{itemize}[<+->]
		\item Kein Benutzer und keine Gruppe von Benutzern kontrolliert die Geräte anderer Benutzer und somit einen Teil des Netzes
		\item Betreiber haben allerdings die Wahl einer Fernwartung ihrer Knoten zuzustimmen.
		\item Eingriffe in die Knoten, z.B. Firmwareaktualisierungen oder Fernwartung, geschieht nur mit explizitem Einverständnis des jeweiligen Betreibers
	\end{itemize}
\end{frame}

\begin{frame}
	\frametitle{Vision: unkommerzieller Netzbetrieb}
	
	\begin{itemize}[<+->]
		\item Das Netz ist kostenlos zugänglich.
	\end{itemize}
\end{frame}

\begin{frame}
	\frametitle{Vision: Privatsphäre erhalten}
	
	\begin{itemize}[<+->]
		\item Wir speichern weder Verbindungsdaten noch schneiden wir Datenverkehr mit
		\item Wir speichern keine persönlichen Daten
		\begin{itemize}
			\item mit Ausnahme der von Betreibern explizit zum Betrieb des Netzes freiwillig bereitgestellten Daten wie Kontaktinformationen und geographischen Koordinaten
		\end{itemize}
	\end{itemize}
\end{frame}

\begin{frame}
	\frametitle{Ziele von Freifunk}
	\begin{itemize}[<+->]
		\item das Verständnis von Datennetzen und Netzwerktechnik sowie deren Auswirkungen auf die Gesellschaft zu fördern,
		\item die breite Bevölkerung zu ermutigen, sich aktiv an der Forschung, Entwicklung und Bildung von freien (drahtlosen) Netzen zu beteiligen,
		\item 	alle unsere Fachkenntnisse und Implementierungen öffentlich zugänglich zu machen und anderen zu ermöglichen, darauf auf zu bauen
		\item und sich am politischen Prozess zu beteiligen Maßnahmen, um die rechtlichen Voraussetzungen für freie Netze zu schaffen.
	\end{itemize}
\end{frame}


\section{Freifunk Darmstadt}


%Radiostationen, Filmkanäle, Telefoniedienste, Blogs,

%\begin{frame}{Warum Freifunk?}
%\begin{center}\includegraphics[width=0.8\textwidth]{why}
%\vfill
%\pause We need private, free communication!

%Note for translation: private in terms of privacy, not ownership! :P )
%\end{center}
%\end{frame}





\begin{frame}{Was wollen wir für Darmstadt?}
\vfill
Etwas ausführlicher\ldots
\begin{itemize}
\pause\item Netzwerk erweitern\pause, aber nicht nur \ldots
\pause\item Dienste anbieten
\pause\item Mit anderen Communities verbinden (ICVPN)
\pause\item Experimente mit unterschiedlichen Protokollen, Software und Hardware
\end{itemize}
\begin{center}
\vfill
\includegraphics[width=0.4\textwidth]{images/disassemble}
\end{center}
\vfill
\end{frame}

\section{Nächste Schritte}
\begin{frame}{Nächste Schritte}
\vfill
\begin{itemize}
\pause\item potentielle Unterstützer kontaktieren:
	\begin{itemize}
		\pause\item Einwohner
		\pause\item Geschäfte
		\pause\item Stadt
		\pause\item politische Gruppen
		\pause\item Unis
		\pause\item Studentenwohnheime, \ldots
	\end{itemize}
\pause\item Vorträge und Workshops organisieren
\pause\item Technik ausbauen und redundant machen
\pause\item Sponsoren finden für Technik und öffentliche APs
\end{itemize}
\vfill
\end{frame}

\section{Rechtliche Aspekte und Risiken}
\begin{frame}{Rechtliche Aspekte und Risiken}
\begin{Row}
\begin{Cell}{1}
\vspace{0.1cm}
\includegraphics[width=3.7cm]{images/recht}
\end{Cell}
\begin{Cell}{2}
\vspace{1cm}
Überblick:
\begin{itemize}
\pause \item Störerhaftung
\pause \item Risiken fürs eigene Netz
\pause \item Sicherheit im Freifunk-Netz
\pause \item was Freifunk (noch) nicht bietet
\end{itemize}
\end{Cell}
\end{Row}
\end{frame}

\begin{frame}{Störerhaftung}
\begin{itemize}
\pause\item Teilen des eigenen Internetanschlusses ist in Deutschland legal
\begin{itemize}
	\pause\item \textbf{aber:} Mithaftung für illegale Aktivitäten, wenn Täter nicht zu ermitteln ist
\end{itemize}
\pause\item Abmahnkosten sind zu tragen
\vfill
\pause\item aktuelle Lösung: VPN in Länder ohne Störerhaftung
\pause\item Zukunft: selbst Internetanbieter werden
\end{itemize}
\vfill
\centering
\pause \textbf{Fazit:}\\Störerhaftung existiert noch, wir machen es aber sehr schwer,\\den eigentlichen Anschlussinhaber zu finden.

\end{frame}

\begin{frame}{Risiken fürs eigene Netz}
\vfill
\begin{center}
\includegraphics[height=4cm]{images/WR842ND-back}
\end{center}
\begin{itemize}
\pause\item Datenverkehr getrennt
\pause\item eigener Internetzugang nur für VPN zu anderen Knoten im Freifunk-Netz
\pause\item eigenes Netz und Freifunk-Netz sind getrennt, so lange Firmware keine Fehler hat oder gehackt wird
\end{itemize}
\vfill
\end{frame}

\begin{frame}{Sicherheit im Freifunk-Netz}
\begin{itemize}
	\pause\item Freifunk als freies Netz hat wenige Einschränkungen
	\pause\item Menschen können darin auch böse Dinge tun
	\vfill
	\pause\item Daten im Freifunk-Netz: Gesamter Verkehr schwer zu überwachen, aber nicht verschlüsselt
	\pause\item Daten ins Internet: Betreiber der Gateways können überwachen
	\vfill
\pause\item \textbf{Achtung:} WLAN ist nicht verschlüsselt, vertrauliche Daten nur mit Ende-zu-Ende-Verschlüsselung (z.B. https) übertragen
\vfill
\end{itemize}
\centering
\textbf{Fazit:}\\Zentrale Überwachung und Zensur deutlich schwieriger \\ als im Internet
\end{frame}


\begin{frame}{Was Freifunk (noch) nicht bietet}
\vfill
\begin{itemize}
\pause\item Schutz vor Trojanern, Phishing und anderen Gefahren des freien Datenverkehrs
\begin{itemize}
\pause\item[$\rightarrow$] wird es nicht geben, sonst ist es kein freier Datenverkehr mehr
\end{itemize}
\vfill
\pause\item Verschlüsselung gibt es nicht auf allen Teilen der Infrastruktur
\vfill
\pause\item Experimente mit verschlüsseltem Meshing und WLAN AP
\end{itemize}
\vfill
\end{frame}

\begin{frame}{Q~\&~A}
\vfill
\centering
\includegraphics[width=0.7\textwidth]{images/irl_router}
\vfill
\end{frame}


\end{document}
