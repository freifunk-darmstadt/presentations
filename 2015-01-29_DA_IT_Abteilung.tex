\documentclass{beamer}
\usepackage[utf8]{inputenc}
\usetheme{default}
\usecolortheme{dove}
\usepackage{textpos}
\usepackage{grid-system}
\usepackage[official]{eurosym}


% po4a: environment frame
% po4a: environment Row
% po4a: environment Cell


\title{Freifunk Darmstadt}
\author{\includegraphics[width=4cm]{images/logo}}
%\institute[Inst.]{Freifunk Darmstadt}
\date{29. Januar 2014}

\begin{document}

\begin{frame}
\maketitle
\end{frame}

\addtobeamertemplate{frametitle}{}{%
\begin{textblock*}{100mm}(0.93\textwidth,-0.5cm)
\includegraphics[height=1cm]{images/logo}
\end{textblock*}}

\begin{frame}{Outline}
\tableofcontents
\end{frame}

\section{Einleitung}
\begin{frame}{Freifunk in Deutschland}
\vfill
\begin{center}
\includegraphics[height=0.6\textheight]{images/map}$\;$
\includegraphics[height=0.6\textheight]{images/moehne}
\end{center}
\vfill
\end{frame}

\section{Projektbeschreibung}
\begin{frame}{Projektbeschreibung}
\vfill
\begin{center}
\includegraphics[height=0.7\textheight]{images/meshing}
\end{center}
\vfill
\end{frame}

\section{Aktueller Stand}
\begin{frame}{Aktueller Stand}
\vfill
\begin{center}
\includegraphics[height=0.75\textheight]{images/2015-01-26-map}$\;$
\vfill
ca. 80 Accesspoints
\end{center}
\end{frame}

\begin{frame}{Aktueller Stand - Darmstadt}
\vfill
\begin{center}
\includegraphics[height=0.75\textheight]{images/2015-01-26-darmstadt}$\;$
\vfill
\end{center}
\end{frame}

\begin{frame}{Aktueller Stand - K6}
\vfill
\begin{center}
\includegraphics[height=0.75\textheight]{images/2015-01-26-wohnart3}$\;$
\vfill
\end{center}
\end{frame}

\section{Ausbauplan}
\begin{frame}{Ausbauplan}
\vfill
\begin{itemize}
	\item öffentliche Plätze, Staatstheater, Haltestellen, Krankenhäuser
	\item Hotels, Gaststätten, öffentliche Einrichtungen
	\item Parks (z.B. Herrengarten, Prinz-Emil-Garten)
	\item private Wohnungen, Studentenwohnheime
	\item hochgelegene Plätze für Richtfunk (z.B. Langer Ludwig, Kirchtürme, Hochzeitsturm, h$\_$da Hochhaus)
\end{itemize}
\vfill
\pause
\begin{itemize}
	\item Anlieger stellen der Freifunk-Community Standorte zur Verfügung und betreiben eigene Freifunk-Router
	\item Durchführung von Informationsveranstaltungen und Workshops über Freifunk und den sicheren Umgang damit
\end{itemize}
\vfill
\end{frame}

\section{Verwendete Router-Hardware}
\begin{frame}{Verwendete Router-Hardware}
Handelsübliche Modelle im 2.4GHz- und 5GHz-Band
\vfill
Für den Heimbedarf oder kleinere öffentliche Bereiche:
\begin{itemize}
\item bis zu 15-25 Clients pro Gerät
\item 30-70\euro{}
\end{itemize}
\vfill

Größere Inneninstallationen:
\begin{itemize}
\item bis zu 100 Clients pro Gerät und Frequenzband
\item ca. 250\euro{}
\end{itemize}
\vfill

Außeneinsatz:
\begin{itemize}
\item bis zu 100 Clients pro Gerät und Frequenzband
\item Kosten abhängig von Frequenzband, Geschwindigkeit und Antennentyp, 100-600\euro{}
\end{itemize}
\vfill
\end{frame}

\section{Beispiel Luisenplatz}
\begin{frame}{Beispiel: Luisenplatz}
Wie sieht die Leistungsfähigkeit der technischen Lösung am Beispiel Luisenplatz aus?
\end{frame}

\section{Anforderungen an die Stadt}
\begin{frame}{Anforderungen an die Stadt}
\vfill
\begin{center}
\includegraphics[height=0.4\textheight]{images/setup}$\;$
\vfill
\end{center}
\begin{itemize}
\item Bereitstellung von Standorten/Montageflächen
\item Bereitstellung von Strom und Internetanbindung/Richtfunk
\item Erwerb oder Sponsoring der notwendigen Hardware
\item standortabhängig Durchführung fachgerechter Montagearbeiten
\end{itemize}
\vfill
\end{frame}

\section{Organisation}
\begin{frame}{Organisation}
\vfill
\begin{itemize}
\item das Freifunk-Kernteam ist zuständig für
	\begin{itemize}
	\item Netzwerkinfrastruktur
	\item Firmwareaktualisierung
	\item Communitymanagement
	\item Support
	\end{itemize}
\vfill
\item die Freifunk-Community besitzt die Knoten und ist für deren Betrieb verantwortlich
\end{itemize}
\vfill
\end{frame}

\begin{frame}{Organisation - Infrastruktur}
\vfill
\begin{itemize}
	\item Multiple VPN-Endpunkte
	\item Redundante Internetanbindung
	\item kontinuierliches Monitoring aller kritischen Systeme
\end{itemize}
\vfill
Zeit bis zum Failover bei Ausfall einer Internetanbindung: max. 60s
\vfill
\end{frame}

\begin{frame}{Organisation - Firmware}
\vfill
\begin{itemize}
	\item Gluon Framework, basierend auf OpenWrt
	\item Entwicklung durch deutschlandweite Community
	\item integrierter Updatemechanismus
	\begin{itemize}
		\item Integrität und Authentizität durch kryptographische Signatur sichergestellt
	\end{itemize}
\end{itemize}
\vfill
Ausrollen neuer Firmware-Releases binnen 24 Stunden
\vfill
\end{frame}
	
\begin{frame}{Organisation - Communitymanagement}
\vfill
\begin{itemize}
\item Regelmäßige Informationsveranstaltungen
\item Support bei unseren Treffen und Online
\item Öffentlichkeitsarbeit
\end{itemize}
\vfill
\end{frame}

\end{document}
